\documentclass{scrreprt}
\usepackage{enumitem}
\usepackage{listings}
\usepackage{underscore}
\usepackage[ddmmyyyy]{datetime}
\renewcommand{\dateseparator}{--}
\usepackage[bookmarks=true]{hyperref}
\usepackage[utf8]{inputenc}
\usepackage[english]{babel}
\usepackage{url}
\usepackage{graphicx}
\usepackage{tabu}


\newenvironment{enum}
{\begin{enumerate}[label*=\arabic*.][resume]}
{\end{enumerate}}

\hypersetup{
    bookmarks=false,    % show bookmarks bar?
    pdftitle={Test Cases Document},    % title
    pdfauthor={Preet Patel, Matt Horger, John Zlotek},                     % author
    pdfsubject={TeX and LaTeX},                        % subject of the document
    pdfkeywords={TeX, LaTeX, graphics, images}, % list of keywords
    colorlinks=true,       % false: boxed links; true: colored links
    linkcolor=blue,       % color of internal links
    citecolor=black,       % color of links to bibliography
    filecolor=black,        % color of file links
    urlcolor=purple,        % color of external links
    linktoc=page            % only page is linked
}%
\def\myversion{2.0.0 }
\date{}
%\title
\usepackage{hyperref}
\begin{document}

\begin{flushright}
    \rule{16cm}{5pt}\vskip1cm
    \begin{bfseries}
        \Huge{Software Validation \& Test Cases\\Document}\\
        \vspace{1.0cm}
        for\\
        \vspace{1.0cm}
        Checkers\\
        \vspace{1.5cm}
        \LARGE{Version \myversion}\\
        \vspace{1.5cm}
        Prepared by:\\
    John Zlotek\\
    Matt Horger\\
    Jake Carfagno\\
    Preet Patel\\
        \vspace{1.9cm}
        Team: \textbf{Big Chungus}\\
        \vspace{1cm}
        \today\\
    \end{bfseries}
\end{flushright}

\tableofcontents

\chapter*{Revision History}

\begin{center}
    \begin{tabular}{|c|c|c|c|}
        \hline
        Name & Date & Reason For Changes & Version\\
        \hline
        1.0.0 & \formatdate{12}{8}{19} & Initial Draft & pp534\\
       \hline
        2.0.0 & \formatdate{14}{8}{19} & Final Draft and Revisions & mh3294\\
        \hline
    \end{tabular}
\end{center}

\chapter{Introduction}

\section{Purpose of Document}
The purpose of this document to document and demonstrate that our Checkers game design and functionality
aligns with the requirements outlined in the referenced requirement document. This document includes validation and testing
of all possible scenarios by defining accepting states and noting the actual outcome of each respective case.

\section{Scope of Document}
The scope of this document encompasses the three phases that are each unique parts in our overall game procedure. We chose to separate our tests into three phases because they are each independent in that they encapsulate similar tests, yet are dependent on each other via a linking action or procedure.
This document will try to justify that our design has been tested and meets the entire specifications of our requirements, and will report any bug that exists within our testing suite.

\section{References}
All required references can be found in the preceding document at the terminal end. Most of these references will point towards are design and requirements document, which have been previously submitted.

\chapter{Testing Environment}
This section has brief information about the system environment where the test suite was performed and the
information of the tester.

\section{Developer's Environments}

\begin{center}
	\begin{tabular}{|l|l|l|l|l|l|}
	\hline
  	Machine Name & OS & Client/Server & JRE & Person & Date \\
 	\hline
          patel_laptop & MacOS & Client & Java 12 & Patel,P & \formatdate{12}{8}{19}\\
          \hline
          horger_desktop & Windows 10 & Server & Java 12 & Horger,M & \formatdate{14}{8}{19}\\
          \hline
          zlotek_laptop & Ubuntu and Arch Linux & Server & Java 12 & Zlotek,J & \formatdate{19}{8}{19}\\
          \hline
          carfagno_desktop & Windows 10 & Client & Java 12 & Carfagno, J & \formatdate{19}{8}{19}\\
         \hline

	\end{tabular}
\end{center}

\chapter{Testing Environment Setup and Prerequisites}

Prerequisite
\begin{itemize}
  \item This game has implemented a Java Swing GUI for interaction. Java 12.0 is needed on the target system in order for the user to properly see the UI without any issues.
  \item Just for testing purpose, our program can be tested on a single, localhost machine. For multiplayer games with other devices, a stable Internet connection is needed.
  \item A Drexel Account is required to interact with the lobby structure to satisfy the tests conditions.
\end{itemize}

\chapter{Test Cases}

This chapter is divided into four sections, each respective to each phase of the game. The first phase includes the launching portion of the game, as well as interacting with the key menus to launch lobbies. The second phase tests the validation and logic of gameplay. The Third phase, while small, tests the functionality of multiple sessions running concurrently. Finally, the fourth phase tests win conditions and outlying cases.

\section{Test Phase 1: Starting Game}

\subsection{Description}
This section covers the testing of the initial phase of launching the program, creating a lobby, joining a lobby and starting a new game. This section relies heavily on UI interaction with our Swing menus.
\subsection{Prerequisites for this test case} An internet connection with a firewall rule allowing traffic on all ports, to be certain, is required to this phase. Localhost also works just fine, as long as two sessions are able to run on the target machine with no drawbacks.
\subsection{Scenario}

\begin{tabu} to \textwidth {| c | X | X | X | X | X |}
\hline
\textbf{Number} & \textbf{Case} & \textbf{Expectation} & \textbf{Actual Outcome} & \textbf{Steps} & \textbf{Requirement Refrence}\\ \hline
1 & Launch the game & GUI of Lobby menu should be launched &  &  & \\ \hline
2 & Create a session & Create a session ID &  &  &R4.1 \\ \hline
3 & Join new game & Once lobby selected from loby menu, player should be able to join that session &  &  &R4.2 \\ \hline
4 & Create multiple session Max 5 & No more then 10 session should be allowed &  &  &L2.2\\ \hline
5 & Start 5 game succeffuly with 10 player & There should be 10 gameplay going on a same time on a server &  &  &L2.3\\ \hline
6 & Start 5 game succeffuly with 11 player & 21th player should not able to join any session. &  &  &L2.3\\ \hline
7 & After starting game with two player & GUI should have 8X8 board with randomly assigned color &  &  &G1 \\ \hline
\end{tabu}
\newpage


\section{Test Phase 2: Gameplay validation}

\subsection{Description}
This section covers the testing of the gameplay validation, it covers the test of moves which are valid, moves are invalid, valid jump, invalid jump and such.

\subsection{Prerequisites for this test case}
A valid game session running - single session at least.

\subsection{Scenario}
\begin{tabu} to \textwidth {| c | X | X | X | X | X |}
\hline
\textbf{Number} & \textbf{Case} & \textbf{Expectation} & \textbf{Actual Outcome} & \textbf{Steps} & \textbf{Requrienment Refrence}\\ \hline
1 & Black piece first move & Only player with black piece should make move &  &  &G2\\ \hline
2 & White piece invalid first move & If Player with white piece makes the move then it should be discarded and the piece should come back to the original position &  &  &G2\\ \hline
3 & Valid Move - Non crowned Piece & Can move diagonally in either direction and peice stays in new square &  &  &G5\\ \hline
4 & Valid single jump to capture - Non crowned Piece & Only jump and capture the opponent piece if there is an empty square above that targeted piece. &  &  &G5 \\ \hline
\end{tabu}
\newpage
\begin{tabu} to \textwidth {| c | X | X | X | X | X |}
\hline
\textbf{Number} & \textbf{Case} & \textbf{Expectation} & \textbf{Actual Outcome} & \textbf{Steps} & \textbf{Requrienment Refrence}\\ \hline
5 & Valid Multiple jump to capture - Non crowned Piece & If jump made over multiple piece of opponent considering test case 4 then it should be valid. &  &  &G5 \\ \hline
6 & Invalid Move - Non crowned Piece & If piece move in horizontal or vertical direction then it should come back to original position &  &  &G5 \\ \hline
7 & Invalid Move - Non crowned Piece & If tried move backward diagonally then it should be discarded and  it should come back to original position &  &  &G5 \\ \hline
8 & Invalid single jump to capture - Non crowned Piece & if there is not space to place the piece after the jump to capture then it should come back to original position &  &  &G5 \\ \hline
9 & Invalid single jump to capture - Non crowned Piece & If jump made over players own piece then  &  &  &G5 \\ \hline
\end{tabu}
\newpage
\begin{tabu} to \textwidth {| c | X | X | X | X | X |}
\hline
\textbf{Number} & \textbf{Case} & \textbf{Expectation} & \textbf{Actual Outcome} & \textbf{Steps} & \textbf{Requrienment Refrence}\\ \hline

10 & Turning into crown piece & If either player piece reaches to the 10th row from player direction then, piece should turn into a crowned piece &  &  &G6 \\ \hline
11 & Valid Move - Crowned Piece & It can move in any four diagonal direction &  &  &G6 \\ \hline
12 & Valid single jump to capture - Crowned Piece & It can jump in any direction considering the test case 4 is valid with each jump &  &  &G6 \\ \hline
13 & Valid Multiple jump to capture - Crowned Piece & it can jump in any direction considering the test case 12 is valid with each jump &  &  &G6 \\ \hline
14 & Invalid Move - Crowned Piece & If moved in horizontal or vertical then piece should return in original direction &  &  &G6 \\ \hline
15 & Invalid single jump to capture - Crowned Piece & If jumped is made over own piece then it discards it and peice should return in original direction &  &  &G6 \\ \hline
16 & If crowned piece placed in the 10th row again & The crown piece state should not be affected in any way &  &  & G6\\ \hline
\end{tabu}


\section{Test Phase 3: Multiple}

\subsection{Description}
This section covers the testing of the proper game performance of multiple session running the same time.

\subsection{Prerequisites for this test case}
More than one game session running to test this.

\subsection{Scenario}
\begin{tabu} to \textwidth {| c | X | X | X | X | X |}
\hline
\textbf{Number} & \textbf{Case} & \textbf{Expectation} & \textbf{Actual Outcome} & \textbf{Steps} & \textbf{Requrienment Refrence}\\ \hline
1 & Session activity only between two connected player & Only changes should be done and seen between two connected player, it should not interfier other session game play &  &  &1.1 \\ \hline
\end{tabu}

\section{Test Phase 4: Ending the game}

\subsection{Description}
This section covers the testing of the scenario when the game ends and possibilities of further action.

\subsection{Prerequisites for this test case}
Atleast one game running

\subsection{Scenario}
\begin{tabu} to \textwidth {| c | X | X | X | X | X |}
\textbf{Number} & \textbf{Case} & \textbf{Expectation} & \textbf{Actual Outcome} & \textbf{Steps} & \textbf{Requrienment Refrence}\\ \hline
1 & Winning Jump & If the valid jump is made over the last piece of the opponet then player should be decalred winner &  &  &6.2.1.1 \\ \hline
2 & Winning Player Screen & Player should be informed with the message  &  &  &6.2.1.1 \\ \hline
3 & Lossing Player Screen & Player should be informed with the message  &  &  &6.2.1.1 \\ \hline
4 & Player 1 wants to quit & Once player 1 click on quit it should send him back to lobby page &  &  &6.2.1.2 \\ \hline
5 & Player 1 left the game & Player 2 will be informed that he/she won the game and send back to lobby page &  &  &6.2.1.2 \\ \hline
6 & After game is fineshed if player 1 choose to rematch and Player 2 also do the same & If Player 2 also click on the rematch then new game should get going on same session &  &  &6.2.1.2 \\ \hline
7 & After game is fineshed if player 1 choose to rematch and Player 2 quit & if player 2 click on the quit then both player should be returned to lobby menu &  &  &6.2.1.2 \\ \hline
\end{tabu}

\chapter{Appendix}
\section{Glossary}
Tux - Place where server is hosted, Tux is offered by Drexel University.


\begin{thebibliography}{9}

\bibitem{checkersFoundation}
  The American Checker Foundation,
  \textit{USA Checkers},
  https://www.usacheckers.com/,
  2019.

\bibitem{historyCheckers}
W.J. Rayment,
\textit{History of Checkers or Draughts},
http://www.indepthinfo.com/checkers/history.shtml,
2004.

\end{thebibliography}


\end{document}
